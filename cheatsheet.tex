%%%%%%%%%%%%%%%%%%%%%%%%%%%%%%%%%%%%%%%%%
% Cheatsheet
% LaTeX Template
% Version 1.0 (12/12/15)
%
% This template has been downloaded from:
% http://www.LaTeXTemplates.com
%
% Original author:
% Michael Müller (https://github.com/cmichi/latex-template-collection) with
% extensive modifications by Vel (vel@LaTeXTemplates.com)
%
% License:
% The MIT License (see included LICENSE file)
%
%%%%%%%%%%%%%%%%%%%%%%%%%%%%%%%%%%%%%%%%%

%----------------------------------------------------------------------------------------
%	PACKAGES AND OTHER DOCUMENT CONFIGURATIONS
%----------------------------------------------------------------------------------------

\documentclass[11pt]{scrartcl} % 11pt font size

\usepackage[utf8]{inputenc} % Required for inputting international characters
\usepackage[T1]{fontenc} % Output font encoding for international characters

\usepackage[margin=0pt, landscape]{geometry} % Page margins and orientation

\usepackage{graphicx} % Required for including images

\usepackage{color} % Required for color customization
\definecolor{mygray}{gray}{.75} % Custom color

\usepackage{url} % Required for the \url command to easily display URLs

\usepackage[ % This block contains information used to annotate the PDF
colorlinks=false,
pdftitle={Personnal Cheatsheet},
pdfauthor={Lqp1},
pdfsubject={Compilation of useful shortcuts},
pdfkeywords={Dev, Vim}
]{hyperref}

\setlength{\unitlength}{1mm} % Set the length that numerical units are measured in
\setlength{\parindent}{0pt} % Stop paragraph indentation

\renewcommand{\dots}{\ \dotfill{}\ } % Fills in the right amount of dots

\newcommand{\command}[2]{#1~\dotfill{}~#2\\} % Custom command for adding a shorcut

\newcommand{\sectiontitle}[1]{\paragraph{#1} \ \\} % Custom command for subsection titles

%----------------------------------------------------------------------------------------

\begin{document}

\begin{picture}(297,210) % Create a container for the page content

%----------------------------------------------------------------------------------------
%	TITLE SECTION
%----------------------------------------------------------------------------------------

\put(10,200){ % Position on the page to put the title
\begin{minipage}[t]{210mm} % The size and alignment of the title
\section*{Compilation of useful shortcuts} % Title
\end{minipage}
}

%----------------------------------------------------------------------------------------
%	FIRST COLUMN SPECIFICATION
%----------------------------------------------------------------------------------------

\put(10,180){ % Divide the page
\begin{minipage}[t]{80mm} % Create a box to house text

%----------------------------------------------------------------------------------------
%	HEADING ONE
%----------------------------------------------------------------------------------------

\sectiontitle{Vim}

\command{:ab x y}{Abbreviation for a -> y}
\command{:una x}{Remove abbreviation}
\command{:split}{Split screen}
\command{:vertical resize -20}{Reduce viewport}
\command{:only}{Just one window}
\command{:ls}{List buffers}
\command{:b}{Select buffer}
\command{:bd}{Remove buffer}
\command{:noh}{Remove highlight from last search}
\command{:g/.../d}{Delete matching lines}
\command{:v/.../d}{Delete not matching lines}
\command{:bufdo vimgrepadd \%}{command for all buffers}
\command{:vimgrep :lvimgrep}{grep in file to :cw or :lw}
\command{:cexpr []}{empty quickfix list}
\command{:reg}{List registers}
\command{"x5yw}{Specify buffer}
\command{mx 'x}{save/go to mark x}
\command{K}{Call man under cursor}
\command{c/pattern}{action to next search}
\command{J}{join lines}
\command{gf}{go to file under cursor}
\command{za}{Fold / unfold}
\command{M,H,L,zz}{Middle, Top, Bottom, go to center}
\command{cM, cc}{Change M, or change line}
\command{ytx, yfx, yFx, yTx}{copy til/to after/before}
\command{V}{Select lines}
\command{q, @}{Record / Start macro}
\command{C-a, C-x}{Inc/Dec int under cursor}
\command{gi gv}{Reinsert, reselect}


%----------------------------------------------------------------------------------------
%	HEADING TWO
%----------------------------------------------------------------------------------------



\end{minipage} % End the first column of text
} % End the first division of the page

\put(95,180){ % Divide the page
\begin{minipage}[t]{80mm} % Create a box to house text

\sectiontitle{Vim, continued}

\command{<CTRL> O}{Ex mode for 1 command}
\command{<C-X><C-L>}{Complete line}
\command{<CTRL-W> +}{Increase viewport size}
\command{<CTRL-W> =}{Split viewport size}
\command{<CTRL-K> XX}{Enter digraph}
\command{<C-R><C-W>}{Read work in cmd toolbar}
\command{<C-R><REG>}{Read from buffer; = means calc}
\command{C-w}{Delete word before}
\end{minipage} % End the second column of text
} % End the second division of the page

%----------------------------------------------------------------------------------------
%	THIRD COLUMN SPECIFICATION
%----------------------------------------------------------------------------------------

\put(180,180){ % Divide the page
\begin{minipage}[t]{80mm} % Create a box to house tex

%----------------------------------------------------------------------------------------
%	IMPORTANT FILES
%----------------------------------------------------------------------------------------


\sectiontitle{Dev}

\command{dpkg-scanpackages . /dev/null | gzip -9c > Packages.gz}{Local Repo}
\command{git diff --no-prefix > x.patch}{Create patch from diff}
\command{git format-patch -1 <SHA>}{Create patch from single commit}
\command{patch -p0 < x.patch}{Apply patch}
\command{git apply --stat x.patch}{Apply patch from git}

\sectiontitle{Emacs}
\command{M-x}{Start command}
\command{C-x C-f}{Find file}
\command{C-x C-s}{Save}
\command{C-x C-c}{Quit}
\command{C-x 2/3}{Split vert/horiz}
\command{C-x o}{Change viewport}
\command{C-x b}{List buffers}
\command{C-x 0}{Close current viewport}
\command{C-s}{Incremental search}
\command{C-g}{Cancel current command}
\command{C-M-s}{Incremental Regexp search}
\command{M-x shell/dired}{Start shell, start dired}
\command{C-x C-f /login@host:/file}{Tramp edit remote file, use pipe to chain}

%----------------------------------------------------------------------------------------
%	FOOTNOTE
%----------------------------------------------------------------------------------------

\vspace{\baselineskip}
\linethickness{0.5mm} % Thickness of the footer line
{\color{mygray}\line(1,0){30}} % Print the line with a custom color

\footnotesize{

}

%----------------------------------------------------------------------------------------

\end{minipage} % End the third column of text
} % End the third division of the page
\end{picture} % End the container for the entire page

%----------------------------------------------------------------------------------------

\end{document}
